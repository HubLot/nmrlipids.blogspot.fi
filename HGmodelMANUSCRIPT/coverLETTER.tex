\documentclass[11pt]{letter}
\usepackage{graphicx,baskervald,microtype}
\usepackage{hyperref,amsmath,xcolor}
\usepackage{pgfplots,marginnote}
\usepackage[top=0.5in,bottom=0.5in,left=1in,right=1in]{geometry}
%\newgeometry{margin=2.5cm}
\newcommand{\omamargin}[1]{\marginnote{\textbf{#1}}[7pt]}
\begin{document}
\reversemarginpar
\pagestyle{empty}
\noindent O. H. Samuli Ollila \\
\noindent Helsinki Biophysics and Biomembrane Group\\
\noindent Department of Biomedical Engineering and Computational Science\\
\noindent P.O. Box 12200, FI-00076 Aalto University, Espoo, Finland\\
\noindent samuli.ollila@aalto.fi, Tel. +358503746963, Fax. +358 9 470 23182 \\


%THESE ARE COPYPASTED FROM THE JACS AUTHOR GUIDELINES AND I HAVE TRIED TO ANWER THESE
%
%Articles of high scientific quality, originality, significance, and conceptual novelty that
%are of interest to the wide and diverse contemporary readership
%of JACS will be given priority for publication.
%
%it is required that (a) the level of theory and methodology employed must be adequate for the
%problem at hand, and (b) the manuscript must provide significant chemical insights or have
%substantial predictive value.

Dear Editor,

Please find attached to this message a manuscript entitled `Towards atomistic resolution structure of phosphatidylcholine 
headgroup and glycerol backbone at different ambient conditions' by A. Botan et al., which we would like to submit 
for consideration of publication in \textit{Journal of Physical Chemistry} (JPC). In the manuscript a large amount of
existing experimental Nuclear Magnetic Resonance (NMR) data is reviewed and combined with an exceptionally comprehensive 
collection of molecular dynamics (MD) simulation data in order to understand the atomistic resolution structure 
of biologically abundant phospholipid molecules and their assemblies.

We believe that the manuscript is interesting to the wide and diverse contemporary audience of
JPC as it demonstrates how the combination of experimental NMR data with MD simulations can 
be successful in resolving the atomistic resolution structure of biomolecules in various biologically 
relevant conditions. The approach is demonstrated for lipids, however, the extension to, e.g., membrane 
proteins is straightforward. Such an atomistic resolution of biomolecular structure significantly advances 
the understanding in fundamental chemistry, molecular biology, and also supports the development of applications, 
for example new biomaterials or drug delivery systems. 

In addition to its main message, the manuscript also provides an extensive comparison on the structural quality 
of the numerous widely used computational lipid models. This is crucial information 
for the large group scientists who are producing or reading the literature involving computational lipid models.

The work has been conducted by using a novel open collaboration concept, and all the scientific contributions
are made publicly through the blog-based project web page located at \texttt{nmrlipids.blogspot.fi}. 
Thus all the scientific content related to this work (including raw data, discussions, manuscript drafts, etc.)
has been publicly available during the whole course of the project through this blog. This material and the possibility to 
discuss it on the blog page will remain available also after the publication. This approach has significantly 
increased and will further increase the scientific quality and impact of the work. In addition, we have shared 
all the simulation raw data in a format that allows easy reanalysis and therefore reuse of the data for other purposes. 
In addition to the scientific content, we believe that the success of this fresh approach in the field of chemistry 
is highly interesting to the audience of JPC.

The usage of molecular dynamics simulations and open collaboration are adequate for the
problem at hand. Indeed, such simulations are currently the most straightforward method to 
construct atomistic resolution structures to interpret NMR experiments, but
enormous amount of simulation work was required for this work. The only practical
solution to collect these data was the open collaboration concept.
As result, we have produced original chemical insight for the atomistic resolution 
structures of phospholipids in various biologically relevant conditions.

We hope that you can consider this manuscript to be published as an article in the Journal of Physical Chemistry. \\

Sincerely yours,

On behalf of the authors,

O. H. Samuli Ollila

\end{document}
